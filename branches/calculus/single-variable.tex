\section{Single Variable Calculus}

\subsection{Limits}
	\begin{itemize}
		\item Squeeze Theorem
			\begin{equation}
				\begin{cases}
					g(x) \leq f(x) \leq h(x) \\
					\lim_{x \rightarrow a} g(x) = \lim_{x \rightarrow a} h(x) = L
				\end{cases} \\
				\tq \lim_{x \rightarrow a} f(x) = L
			\end{equation}
		\item Fundamental Trigonometric Limit
			\begin{equation}
				\lim_{x \rightarrow 0} \frac{\sin x}{x} = 1
			\end{equation}
		\item Fundamental Exponential Limit
			\begin{equation}
				\lim_{x \rightarrow 0} (1+x)^{\frac{1}{x}} = \lim_{x \rightarrow \infty} \left( 1+ \frac{1}{x} \right)^x = \lim_{x \rightarrow - \infty} \left( 1+ \frac{1}{x} \right)^x = e
			\end{equation}
	\end{itemize}
\subsection{Differentiation}
	\begin{itemize}
		\item Definition
			\begin{equation}
				f'(x)= \lim_{h \rightarrow 0} \frac{f(x+h) - f(x)}{h} \\
				f'(a) = \lim_{x \rightarrow a} \frac{f(x)-f(a)}{x-a}
			\end{equation}
			\begin{itemize}
				\item Constant function
					\begin{equation}
						\dfrac{d}{dx} c = 0
					\end{equation}
			\end{itemize}
		\item Derivative of Transcendent Functions
			\begin{itemize}
				\item Sine function
					\begin{equation}
						\dfrac{d}{dx} \sin x = \cos x
					\end{equation}
				\item Cosine function
					\begin{equation}
						\dfrac{d}{dx} \cos x = - \sin x
					\end{equation}
				\item Logarithm function
					\begin{equation}
						\dfrac{d}{dx} \log x = \frac{1}{x}
					\end{equation}
				\item Exponential function
					\begin{equation}
						\dfrac{d}{dx} e^x = e^x
					\end{equation}
			\end{itemize}
		\item Properties
			\begin{itemize}
				\item Sum and difference
					\begin{equation}
						(u+v)'=u'+v'
					\end{equation}
				\item Product
					\begin{equation}
						(u \cdot v)' = u' \cdot v + u \cdot v'
					\end{equation}
				\item Produc with a constant
					\begin{equation}
						(c \cdot u)' = c \cdot u'
					\end{equation}
				\item Quotient
					\begin{equation}
						\left( \frac{u}{v} \right) ' = \frac{u' \cdot v - u \cdot v'}{v^2}
					\end{equation}
					\begin{itemize}
						\item Polynomial function
							\begin{equation}
								\dfrac{d}{dx} x^n = \dfrac{d}{dx} \frac{1}{x^{-n}} = n x^{n-1}
							\end{equation}
						\item Tangent function
							\begin{equation}
								\dfrac{d}{dx} \tan x = \dfrac{d}{dx} \frac{\sin x}{\cos x} = \sec^2 x
							\end{equation}
						\item Secant function
							\begin{equation}
								\dfrac{d}{dx} \sec x = \dfrac{d}{dx} \frac{1}{\cos x} = \sec x \cdot \tan x
							\end{equation}
					\end{itemize}
			\end{itemize}
		\item Chain Rule
			\begin{equation}
				\left[ f(g(x)) \right]' = f'(g(x)) \cdot g'(x)
			\end{equation}
			\begin{itemize}
				\item Exponential function (not natural)
					\begin{equation}
						\dfrac{d}{dx} a^x = a^x \cdot \log a
					\end{equation}
				\item Logarithm of a function
					\begin{equation}
						\dfrac{d}{dx} \log g(x) = \frac{g'(x)}{g(x)}
					\end{equation}
			\end{itemize}
		\item Derivative of The Inverse Function
			\begin{equation}
				\left( f^{-1} \right)' (x) = \frac{1}{f' \left( f^{-1} (x) \right)}
			\end{equation}
			\begin{itemize}
				\item Arcsine function
					\begin{equation}
						\dfrac{d}{dx} \sin^{-1} x = \frac{1}{\sqrt{1-x^2}}
					\end{equation}
				\item Arctangent function
					\begin{equation}
						\dfrac{d}{dx} \tan^{-1} x = \frac{1}{1+x^2}
					\end{equation}
			\end{itemize}
		\item Mean Value Theorem
			\begin{equation}
				\frac{f(b) - f(a)}{b-a} = f'(c)
			\end{equation}
			in which $a < c < b$
	\end{itemize}
\subsection{Applications of Differentiation}
	\begin{itemize}
		\item L'Hospital Rule
			\begin{equation}
				\lim_{x \rightarrow a} \frac{f(x)}{g(x)} = \frac{\lim_{x \rightarrow a} \frac{f(x) - f(a)}{x-a}}{\lim_{x \rightarrow a} \frac{g(x) - g(a)}{x-a}} = \lim_{x \rightarrow a} \frac{f'(a)}{g'(a)}
			\end{equation}
			Cases: $\frac{0}{0}$ and $\frac{\infty}{\infty}$
		\item Infinity Hierarchy
			\begin{equation}
				 \log x < \sqrt[n]{x} < \sqrt{x} < x < x^2 < x^n < e^x < x! < x^x \rightarrow \infty
			\end{equation}
		\item Curve Sketching
			\begin{equation}
				f'>0 \Rightarrow f \textrm{ is increasing} \\
				f'<0 \Rightarrow f \textrm{ is decreasing} \\
				f''>0 \Rightarrow f' \textrm{ is increasing} \Rightarrow f \textrm{ is concave up} \\
				f''<0 \Rightarrow f' \textrm{ is decreasing} \Rightarrow f \textrm{ is concave down}
			\end{equation}
			\begin{itemize}
				\item General Strategy
					\begin{enumerate}
					
						\item Plot discontinuities (especially infinite), endpoints (or $x \rightarrow \pm \infty$), and easy points (optional);
						\item Solve $f'(x)=0$, and plot critical points and values;
						\item Decide whether $f'>0$ or $f'<0$ on each interval between critical points;
						\item Analyse when the curve is concave up ($f'' > 0$) or down ($f'' < 0$), and what is/are the inflection point(s) ($f''(x_0) = 0$); and
						\item Combine everything.
					\end{enumerate}
			\end{itemize}
		\item Linear Approximation
			\begin{equation}
				f(x) \approx f(a) + f'(a) \cdot (x-a) \qquad (x \approx a)
			\end{equation}
			\begin{enumerate}
				\item Sine
					\begin{equation}
						\sin x \approx x \qquad (x \approx 0)
					\end{equation}
				\item Cosine
					\begin{equation}
						\cos x \approx 1 \qquad (x \approx 0)
					\end{equation}
				\item Exponential
					\begin{equation}
						e^x \approx 1+x \qquad (x \approx 0)
					\end{equation}
				\item Logarithm
					\begin{equation}
						\log(1+x) \approx x \qquad (x \approx 0)
					\end{equation}
				\item Sum to the power of $n$
					\begin{equation}
						(1+x)^n \approx 1+n \cdot x \qquad (x \approx 0)
					\end{equation}
			\end{enumerate}
		\item Quadratic Approximation
			\begin{equation}
				f(x) \approx f(a) + f'(a) \cdot (x-a) + \frac{f''(a)}{2} \cdot (x-a)^2 \qquad (x \approx a)
			\end{equation}
			\begin{enumerate}
				\item Sine
					\begin{equation}
						\sin x \approx x \qquad (x \approx 0)
					\end{equation}
				\item Cosine
					\begin{equation}
						\cos x \approx 1 - \frac{x^2}{2} \qquad (x \approx 0)
					\end{equation}
				\item Exponential
					\begin{equation}
						e^x \approx 1+x+\frac{x^2}{2} \qquad (x \approx 0)
					\end{equation}
				\item Logarithm
					\begin{equation}
						\log(1+x) \approx x - \frac{x^2}{2} \qquad (x \approx 0)
					\end{equation}
				\item Sum to the power of $n$
					\begin{equation}
						(1+x)^n \approx 1+n \cdot x + \frac{n(n-1)}{2} x^2 \qquad (x \approx 0)
					\end{equation}
			\end{enumerate}
		\item Taylor's Series
			\begin{equation}
				f(x) \approx f(a) + \sum_{n=1}^{\infty} \frac{(x-a)^n}{n!} f^{(n)}(a) \qquad (x \approx a)
			\end{equation}
			\begin{enumerate}
				\item Sine
					\begin{equation}
						\sin x \approx x - \frac{x^3}{3!} + \frac{x^5}{5!} - \frac{x^7}{7!} + ... \qquad (x \approx 0)
					\end{equation}
				\item Cosine
					\begin{equation}
						\cos x \approx 1 - \frac{x^2}{2!} + \frac{x^4}{4!} - \frac{x^6}{6!} + ... \qquad (x \approx 0)
					\end{equation}
				\item Exponential
					\begin{equation}
						e^x \approx 1+x+\frac{x^2}{2!} + \frac{x^3}{3!} + ... \qquad (x \approx 0)
					\end{equation}
				\item Logarithm
					\begin{equation}
						\log(1+x) \approx x - \frac{x^2}{2} + \frac{x^3}{3} - \frac{x^4}{4} + ... \qquad (x \approx 0)
					\end{equation}
					\begin{equation}
						\log x \approx (x-1) - \frac{(x-1)^2}{2} + \frac{(x-1)^3}{3} - \frac{(x-1)^4}{4} + ... \qquad (x \approx 1)
					\end{equation}
				\item Arctangent
					\begin{equation}
						\tan^{-1} x \approx x - \frac{x^3}{3} + \frac{x^5}{5} - \frac{x^7}{7} + ... \qquad (x \approx 0)
					\end{equation}
			\end{enumerate}
		\item Power Series
			\begin{equation}
				a_0 + a_1x+a_2x^2+a_3x^3+...=\sum_{n=0}^{\infty} a_nx^n
			\end{equation}
			OBS.: converges when $|x|<R$, where $R$ is the radius of convergence.
			\begin{enumerate}
				\item Geometric Series $(R=1)$
					\begin{equation}
						1+x+x^2+x^3+... = \frac{1}{1-x}
					\end{equation}
			\end{enumerate}
	\end{itemize}
\subsection{Integration}
	\begin{itemize}
		\item Definition
			\begin{equation}
				\int_a^b f(x) \, dx = \textrm{Area under the curve} \\
				\int_a^b f(x) \, dx = \lim_{\Delta x_i \rightarrow 0} \sum_{i=1}^n f(x_i) \Delta x_i
			\end{equation}
		\item Properties
			\begin{enumerate}
				\item
					\begin{equation}
						\int_a^b f(x) \, dx = - \int_b^a f(x) \, dx
					\end{equation}
				\item
					\begin{equation}
						\int_a^c f(x) \, dx = \int_a^b f(x) \, dx + \int_b^c f(x) \, dx
					\end{equation}
				\item
					\begin{equation}
						\int_a^b f(x) \, dx \leq \int_a^b g(x) \, dx \quad \leftrightarrow \quad f(x) \leq g(x)
					\end{equation}
				\item
					\begin{equation}
						\int_a^b (f(x) + g(x))dx = \int_a^b f(x) dx + \int_a^b g(x) dx
					\end{equation}
				\item
					\begin{equation}
						\int_a^b cf(x)dx = c \int_a^b f(x) dx
					\end{equation}
				\item
					\begin{equation}
						\int_a^a f(x) dx = 0
					\end{equation}
			\end{enumerate}
		\item Fundamental Theorem of Calculus
			\begin{equation}
				\int_a^b f(x) \, dx = F(b)-F(a) \\
				F(x)=\int_a^x f(t) \, dt \\
				F'(x)=f(x)
			\end{equation}
		\item Antiderivatives (Indefinite Integral)
			\begin{enumerate}
				\item Powers (Polynomials)
					\begin{equation}
						\int x^n \; dx = \frac{x^{n+1}}{n+1} + C \quad \leftrightarrow \quad n \neq -1
					\end{equation}
				\item Trigonometric Functions
					\begin{equation}
						\int \sin x \; dx = - \cos x + C \\
						\int \cos x \; dx = \sin x + C \\
						\int \sec^2 x \; dx = \tan x + C
					\end{equation}
				\item Important Fractions
					\begin{equation}
						\int \frac{dx}{x} = \log |x|+ C \\
						\int \frac{dx}{\sqrt{1-x^2}} = \sin^{-1} x + C \\
						\int \frac{dx}{1-x^2}= \tan^{-1} x + C
					\end{equation}
				\item Others
					\begin{equation}
						\int \log x \; dx = x(\log x - 1) + C
					\end{equation}
			\end{enumerate}
		\item Properties of Some Transcendental Functions
			\begin{enumerate}
				\item $L(x) = \int_1^{x} \frac{dt}{t}$
					\begin{equation}
						L(ab) = L(a)+L(b)
					\end{equation}
				\item $F(x) = \int_0^x e^{-t^2}dt$
					\begin{equation}
						\lim_{x \rightarrow \infty} F(x) = \frac{\sqrt{\pi}}{2}
					\end{equation}
				\item $Li(x) = \int_2^x \frac{dt}{ln \; t}$
					\begin{equation}
						Li(x) \approx \textrm{number of primes} < x
					\end{equation}
			\end{enumerate}
		\item Improper Integrals $\left( f(x) = \frac{1}{x^p} \right)$
			\begin{equation}
				\int_1^{\infty} \frac{dx}{x^p} \rightarrow \infty \qquad \textrm{(diverges if } p \leq 1) \\
				\int_1^{\infty} \frac{dx}{x^p} = \frac{1}{p-1} \qquad \textrm{(converges if } p > 1) \\
				\int_0^1 \frac{dx}{x^p} \rightarrow + \infty \qquad \textrm{(diverges if } p \geq 1) \\
				\int_0^1 \frac{dx}{x^p} = \frac{1}{p-1} \qquad \textrm{(converges if } p < 1)
			\end{equation}
	\end{itemize}
\subsection{Techniques of Integration}
	\begin{itemize}
		\item Integration by Substitution
			\begin{equation}
				\int_a^b f(g(x)) \cdot g'(x) \, dx = \int_{g(a)}^{g(b)} f(t) \, dt
			\end{equation}
		\item Integration by Parts
			\begin{equation}
				\int_a^b uv' \, dx = (uv)|_a^b - \int_a^b u' v \, dx
			\end{equation}
		\item Trigonometric Integration
			\begin{enumerate}
				\item Powers - Easy case (at least one odd exponent) $\leadsto$ use the fundamental formula and substitution
					\begin{equation}
						\sin^2 x + \cos^2 x = 1
					\end{equation}
				\item Powers - Hard case (only even exponents) $\leadsto$ use half-angle formulas or other trigonometric identities
					\begin{equation}
						\cos^2 x = \frac{1+\cos 2x}{2} \\
						\sin^2 x = \frac{1-\cos 2x}{2}
					\end{equation}
					\begin{itemize}
						\item products $\rightarrow$ sums
							\begin{equation}
								2 \cdot \sin m \cdot \cos n = \sin (m+n) + \sin (m-n) \\
								2 \cdot \cos m \cdot \cos n = \cos (m+n) + \cos (m-n) \\
								2 \cdot \sin m \cdot \cos n = \cos (m-n) - \cos (m+n)
							\end{equation}
					\end{itemize}
				\item Tangent
					\begin{equation}
						\int \tan x \; dx = - ln \, ( \cos x) + C
					\end{equation}
				\item Secant
					\begin{equation}
						\int \sec x \; dx = ln \, (\sec x + \tan x) + C
					\end{equation}
			\end{enumerate}
		\item Summary of Trigonometric Substitutions
			\begin{center} \begin{tabular}{|c|c|c|} \hline
				\textbf{If integrand contains} & \textbf{make substitution} & \textbf{to get} \\ \hline
				$\sqrt{a^2-x^2}$ & $x=a \cos \theta$ or $x=a \sin \theta$ & $a \sin \theta$ or $a \cos \theta$ \\ \hline
				$\sqrt{a^2 +x^2}$ & $x=a \tan \theta$ & $a \sec \theta$ \\ \hline
				$\sqrt{x^2-a^2}$ & $x=a \sec \theta$ & $a \tan \theta$ \\ \hline
			\end{tabular} \end{center}
		\item Partial Fractions (common cases)
			\begin{equation}
				\frac{p(x)}{(x-a)(x-b)} = \frac{A}{x-a} + \frac{B}{x-b} \\
				\frac{p(x)}{(x-a)^2} = \frac{A}{x-a}+\frac{B}{(x-a)^2} \\
				\frac{p(x)}{(x-a)^2(x-b)} = \frac{A}{x-a}+\frac{B}{(x-a)^2}+\frac{C}{x-b} \\
				\frac{p(x)}{x^2+a} = \frac{A x + B}{x^2 + a} \\
				\frac{p(x)}{(x^2+a)^2} = \frac{A x + B}{x^2+a} + \frac{Cx+D}{(x^2+a)^2}
			\end{equation}
	\end{itemize}
\subsection{Applications of Integration}
